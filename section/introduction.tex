Millimeter-wave (mmWave) wireless is fast emerging as the prime
candidate technology for providing multi-Gbps data rates in future wireless networks.
%, WLANs and for fixed wireless access.
% Although
%different parts of the mmWave spectrum (both licensed and unlicensed)
%are currently being explored by the industry, 
The IEEE 802.11ad
standard 
%and its successor 802.11ay specifically target indoor
%WLANs. 802.11ad 
%utilizes the almost 7 GHz of unlicensed spectrum
%centered around 60 GHz to 
provides data rates of up to 6.7 Gbps. 
It
achieves this multi-fold increase over legacy WiFi through
2 GHz-wide channels.
%ization (2.16 GHz, 13.5x wider compared to the 160
%MHz of 802.11ac) and directional antennas.
Multiple commercial devices (both APs and laptops) supporting the standard have been
released over the past few years. 
%and the technology is already making
%its way into smartphones. ~\cite{peraso_802.11ad,asus_802.11ad}.

Nonetheless, communication at mmWave frequencies faces fundamental
challenges due to the high propagation and penetration loss. %Although 
The use of directional transmissions 
%overcomes the severe attenuation, it 
makes
links susceptible to disruption by human blockage and client
mobility. 
%While several recent works have explored ways of dealing with blockage and mobility, e.g., by
%proposing efficient beamsteering techniques, link outages remain a
%major challenge preventing 802.11ad from providing the seamless
%connectivity that legacy WiFi does. 
Even if PHY and MAC layer
improvements may result in faster beam steering and lower re-connection times in the future, any realistic
indoor scenario is expected to contain enough dynamism
%, in terms of
%environmental and device mobility, 
to cause a large number of
re-connection events which will hurt application performance and
result in poor user experience.
% This issue is critical as
%applications that are considered prime targets for the adoption of 
%60 GHz technology typically cannot work with unreliable links that fail
%frequently. 
%In addition, due to the mmWave channel characteristics, providing full coverage at 60 GHz 
%is extremely difficult and realistic deployments will most likely have some coverage gaps.
%be ``spotty'', at least in the near future. As 60 GHz APs will be
%with incremental deployment, one can expect to find areas in large
%enterprise settings without 60 GHz coverage.

In this work, we tackle the challenge of supporting the multi-Gbps
throughput provided by the 60 GHz technology while still providing
the reliability of legacy WiFi, which is the key for wide-spread adoption of 60 GHz WLANs.
% Instead of choosing 802.11ac's
%reliability over 802.11ad's performance or vice-versa, we propose
%engaging both technologies simultaneously. 
Using both 802.11ad and
802.11ac interfaces simultaneously not only offers reliability by
providing a fall-back option in case 60 GHz connectivity
becomes unavailable but also allows a client to theoretically obtain
the sum of the data rates offered by the two technologies. 
%Recent commercial 802.11ad NICs~\cite{intel11ad, qca11ad} include an 802.11ac radio on the same chipset. 
Commercial off-the-shelf (COTS)
APs as well as client devices offer tri-band solutions with 2.4, 5, and 60 GHz
interfaces and thus such a multi-band approach is
feasible with existing hardware.
% without any extra hardware.

%However, realizing such a combined 802.11ad-ac interface, made up of
%two rather heterogeneous technologies, comes with it's own set of
%challenges. For instance, such a bundling of interfaces needs to done
%in a manner that is transparent to user applications without requiring
%any changes to them. Also, since most applications rely on TCP as the
%choice of transport protocol, any proposed solution needs to offer
%reliable in-order-delivery end-to-end data transfer service.

%A critical architecture choice 
%towards such a solution 
%is at which layer of the protocol stack to implement such a solution.
We explore Multipath TCP (MPTCP), a
transport layer protocol that uses the 802.11ad and 802.11ac interfaces simultaneously to 
achieve higher throughput when both networks are available and seamlessly falls back to 
802.11ac in an application-transparent manner when the 802.11ad network becomes unavailable. 
MPTCP's design as a transport layer solution
%, as an
%alternative to single-path TCP (SPTCP), 
decouples it both from the
application layer and the IP and MAC layers.
%; thus, a solution
%built around MPTCP can evolve independently from and work with any
%future 802.11 standards in the 5 GHz or 60 GHz space (e.g., 802.11ax
%or 802.11ay, respectively). 
\begin{comment}
In contrast, solutions that try to achieve
a similar functionality at the MAC or lower layers, such as 802.11ad's
Fast Session Transfer (FST)~\cite{80211ad}, are invariably tied to the
802.11 specifications and hence are not future proof. More
importantly, MPTCP by design provides the same guarantees to
applications as single-path TCP (SPTCP) in terms of packet delivery and
includes mechanisms to deal with issues such as packet re-ordering
among different interfaces that would otherwise need to be
addressed by any solution implemented at lower layers of the stack,
thus avoiding an unnecessary duplication of functionality.
\end{comment}

In spite of these attractive features, using MPTCP in multi-band WLANs
is far from straightforward. A large number of recent studies
investigated the performance of MPTCP in scenarios combining WiFi and
cellular (3G/LTE) networks~\cite{raiciu:nsdi2012,saha:mobiwac2017} and showed that the
protocol performs poorly over heterogeneous paths, due to interactions
among out of order TCP packets, scheduling decisions, congestion
control, and limited receiver buffer size.
%Given
%the large bandwidth disparity between sub-6GHz and 60 GHz WiFi, it
%is likely that MPTCP will suffer from similar problems in these
%networks. 
%Recent works that studied MPTCP performance in dual band 5/60 GHz
%WLANs~\cite{sur:mobicom2017,nguyen:vtc2017,saha:mobicomposter2017}
%showed that using MPTCP often results in lower performance than using
%the 802.11ad interface alone, and 
The authors in~\cite{sur:mobicom2017,nguyen:vtc2017} even argue that the two
radios should never be used simultaneously.

In contrast, to the best of our knowledge, our work is the first to 
show that the use of MPTCP is not just a viable but even a very promising solution
towards dual-band 5/60 GHz WLANs. We conduct an experimental
study using COTS APs and laptops with the goal 
%of not only exploring
%the performance of MPTCP in dual-band 802.11ad/ac WLANs, but also
of understanding the causes of the observed performance and uncovering the
pitfalls of the current MPTCP implementation in this new setting.
Our study reveals that 
%a few recently introduced optimizations in the
%MPTCP scheduler, while effective in scenarios
%involving WiFi/cellular interfaces, result in suboptimal performance in 802.11ad/ac
%scenarios, and disabling them yields near
MPTCP can provide optimal throughputs under baseline, static scenarios.
 However, realistic dynamic environments, e.g., involving contention in the 5 GHz
band or blockage of the 802.11ad link, are extremely challenging for
the current MPTCP architecture and result in severe performance
degradation. We then design and implement \name, a new
MPTCP scheduler that addresses the root-cause of the performance
degradation and allows MPTCP to achieve near-optimal throughput under
a wide variety of dynamic use cases. Our evaluation in real
indoor WLAN scenarios shows that \name can achieve up to \textbf{2.5x}
throughput improvement and can reduce application-level delay by several
10s of seconds compared to the default MPTCP.
%We show that how MPTCP's architecture and optimizations introduced to
%it over the years, often results in scenarios where MPTCP performs
%sub-optimally and could even result in performance worse than using
%SPTCP over the fastest subflow.
\if 0
Our work makes the following contributions:
\\
(1) We conduct an extensive measurement study to understand MPTCP
performance in dual-band 802.11ad/ac WLANs with COTS devices under
realistic settings.
%\\
%(2) We develop a comprehensive set of tools to instrument MPTCP
%components, like the queues and scheduler, that help us understand the
%root-cause of the performance issues. We will make these tools
%publicly available for others to further improve MPTCP in new settings.
\\
(2) We design and implement \name, a novel MPTCP scheduler, that
addresses the identified challenges.
\\
(3) We evaluate \name in realistic WLAN settings and show that it
offers significant throughput and delay improvements over the default
MPTCP scheduler.
\fi
